\documentclass[english]{article}

\usepackage{graphicx}
\usepackage{grffile}
\usepackage{babel}



\title{Software Requirements Specification\\
	for the NavUP System\\
	\small Version: 1.0}

\graphicspath{{Pictures/}}

\begin{document}
	\maketitle
	\begin{figure}[!t]
		\includegraphics{up_logo.png}
	\end{figure}
	\pagenumbering{gobble}
	
\begin{center}
\textsc{\Large Team Olive}\\[1cm]


\begin{minipage}{0.4\textwidth}
	\begin{flushleft} \large
		\emph{Student:}\\[0.75cm]
		Keoagile {Dinake}
	\end{flushleft}
\end{minipage}
\begin{minipage}{0.4\textwidth}
	\begin{flushright} \large
		\emph{Student number:} \\[0.75cm]
		u15041744
	\end{flushright}
\end{minipage}


\begin{minipage}{0.4\textwidth}
	\begin{flushleft} \large
		\emph{} \\
		Mark {Klingenberg }
	\end{flushleft}
\end{minipage}
\begin{minipage}{0.4\textwidth}
	\begin{flushright} \large
		\emph{} \\
		
	\end{flushright}
\end{minipage}


\begin{minipage}{0.4\textwidth}
	\begin{flushleft} \large
		\emph{} \\
		Andrew {le Roux}
	\end{flushleft}
\end{minipage}
\begin{minipage}{0.4\textwidth}
	\begin{flushright} \large
		\emph{} \\
		u15311644
	\end{flushright}
\end{minipage}

\begin{minipage}{0.4\textwidth}
	\begin{flushleft} \large
		\emph{} \\
		Hlulani {Makamu}
	\end{flushleft}
\end{minipage}
\begin{minipage}{0.4\textwidth}
	\begin{flushright} \large
		\emph{} \\
		u17223832
	\end{flushright}
\end{minipage}

\begin{minipage}{0.4\textwidth}
	\begin{flushleft} \large
		\emph{} \\
		Banele {Nxumalo}
	\end{flushleft}
\end{minipage}
\begin{minipage}{0.4\textwidth}
	\begin{flushright} \large
		\emph{} \\
		u12201911
	\end{flushright}
\end{minipage}

\begin{minipage}{0.4\textwidth}
	\begin{flushleft} \large
		\emph{} \\
		Letlhogonolo {Tom}
	\end{flushleft}
\end{minipage}
\begin{minipage}{0.4\textwidth}
	\begin{flushright} \large
		\emph{} \\
		u13325095
	\end{flushright}
\end{minipage}
\end{center}

	\newpage

	\tableofcontents
	\newpage

	\pagenumbering{arabic}
	

	\section{Introduction}
		\paragraph\indent
			This document consists of software requirements specification (SRS) for NavUP. The software will help students and guests to explore around UP campus and help them find ideal route to their desired destinations. On this document, the purpose for the software is identified, abbreviations and definitions are supplied.

		\subsection{Purpose}
			\paragraph\indent
				On this document, detailed descriptions of the basic functionalities for NavUP are outlined. Well-designed use case diagrams are supplied to draw in context and depict what the complete product resembles [1]. The final product must locate a venue and best route to the venue, this will help the end-users to arrive earlier than estimated time of arrival (ETA). NavUP is essentially planned to be utilized at UP to orientate both novice end users and freshmen around the campus.

		\subsection{Scope}
			\paragraph\indent
				Project scope constitutes the vision, the need to develop a software, achieving goals and objective of the customer [2]. This software is called NavUP. NavUP is a GIS-based mobile application which will locate user indoors and outdoors, have smarter destination search, quick and intelligent routing and rerouting.  The system will support real-time data analysis (RTDA) to update the user about the road status, different type of road users and give visuals of the road they travelling on.
				
The system will support different levels of user privileges namely end-user level, customer level and developer mode. End-users who use NavUP will have the opportunity to utilize game-like feature where participant steps are recorded, goals are set and rewards will be awarded. The customer can recommend new information to the system that will be useful to end-users according to their interest.  The system will not suggest the most frequent destination according to the other users; it will assume different users have unique destinations.

NavUP will support common navigation functionalities such as current location, time of arrival (TA), directions at the top of the map, scroll, zoom and go back to map view. The software needs both Internet and GPS connection to accurately detect the current location of the user and display the RTDA. Processed data will be stored and maintained in a database, which is located on a campus web-server. The software will support both Android and iOS platforms. 


		\subsection{Definition, Acronyms, and Abbreviations}
			\paragraph\indent
				...

		\subsection{References}
			\paragraph\indent
				\begin{itemize}
				  \item[1] Rachel S. Smith (year) “Writing a Requirements Document For Multimedia and Software Project” 				    CSU Centre for Distributed Learning.
				  \item[2] Ivy Hooks and Lous Wheatcraft (2001) “Scope – Magic” Compliance Automation, Inc
				  \item[3] Karl E. Wiegers (1999) “Writing Quality Requirements” Process Impact www.processimpact.com
				\end{itemize}

		\subsection{Overview}
			\paragraph\indent
				...


	\section{Overall Description}
		\paragraph\indent
			...
		
		\subsection{Product Perspective}
			\paragraph\indent
				...
			
				\subsubsection{System Interfaces}
					\paragraph\indent
						...

				\subsubsection{User Interfaces}
					\paragraph\indent
						...

				\subsubsection{Hardware Interfaces}
					\paragraph\indent
						...

				\subsubsection{Software Interfaces}
					\paragraph\indent
						...

				\subsubsection{Communications Interfaces}
					\paragraph\indent
						...

				\subsubsection{Memory}
					\paragraph\indent
						...

				\subsubsection{Operations}
					\paragraph\indent
						...

				\subsubsection{Site Application Requirements}
					\paragraph\indent
						...


		\subsection{Product Functions}
			\paragraph\indent
				...

		\subsection{User Charateristics}
			\paragraph\indent
				...

		\subsection{Constraints}
			\paragraph\indent
				...

		\subsection{Assumptions and Dependencies}
			\paragraph\indent
				...


	\section{Specific Requirements}
		\paragraph\indent
			...

				\subsection{External Interface Requirements}
					\paragraph\indent
						...

				\subsection{Functional Requirements}
					\paragraph\indent
						...

				\subsection{Preformance Requirements}
					\paragraph\indent
						...

				\subsection{Design Constraints}
					\paragraph\indent
						...
	
				\subsection{Software System attributes}
					\paragraph\indent
						...

				\subsection{Other Requirements}
					\paragraph\indent
						...

		
\end{document}
